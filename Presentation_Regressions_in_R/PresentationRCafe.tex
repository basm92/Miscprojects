% Options for packages loaded elsewhere
\PassOptionsToPackage{unicode}{hyperref}
\PassOptionsToPackage{hyphens}{url}
%
\documentclass[
  10pt,
  ignorenonframetext,
]{beamer}
\usepackage{pgfpages}
\setbeamertemplate{caption}[numbered]
\setbeamertemplate{caption label separator}{: }
\setbeamercolor{caption name}{fg=normal text.fg}
\beamertemplatenavigationsymbolsempty
% Prevent slide breaks in the middle of a paragraph
\widowpenalties 1 10000
\raggedbottom
\setbeamertemplate{part page}{
  \centering
  \begin{beamercolorbox}[sep=16pt,center]{part title}
    \usebeamerfont{part title}\insertpart\par
  \end{beamercolorbox}
}
\setbeamertemplate{section page}{
  \centering
  \begin{beamercolorbox}[sep=12pt,center]{part title}
    \usebeamerfont{section title}\insertsection\par
  \end{beamercolorbox}
}
\setbeamertemplate{subsection page}{
  \centering
  \begin{beamercolorbox}[sep=8pt,center]{part title}
    \usebeamerfont{subsection title}\insertsubsection\par
  \end{beamercolorbox}
}
\AtBeginPart{
  \frame{\partpage}
}
\AtBeginSection{
  \ifbibliography
  \else
    \frame{\sectionpage}
  \fi
}
\AtBeginSubsection{
  \frame{\subsectionpage}
}
\usepackage{lmodern}
\usepackage{amssymb,amsmath}
\usepackage{ifxetex,ifluatex}
\ifnum 0\ifxetex 1\fi\ifluatex 1\fi=0 % if pdftex
  \usepackage[T1]{fontenc}
  \usepackage[utf8]{inputenc}
  \usepackage{textcomp} % provide euro and other symbols
\else % if luatex or xetex
  \usepackage{unicode-math}
  \defaultfontfeatures{Scale=MatchLowercase}
  \defaultfontfeatures[\rmfamily]{Ligatures=TeX,Scale=1}
\fi
\usetheme[]{AnnArbor}
% Use upquote if available, for straight quotes in verbatim environments
\IfFileExists{upquote.sty}{\usepackage{upquote}}{}
\IfFileExists{microtype.sty}{% use microtype if available
  \usepackage[]{microtype}
  \UseMicrotypeSet[protrusion]{basicmath} % disable protrusion for tt fonts
}{}
\makeatletter
\@ifundefined{KOMAClassName}{% if non-KOMA class
  \IfFileExists{parskip.sty}{%
    \usepackage{parskip}
  }{% else
    \setlength{\parindent}{0pt}
    \setlength{\parskip}{6pt plus 2pt minus 1pt}}
}{% if KOMA class
  \KOMAoptions{parskip=half}}
\makeatother
\usepackage{xcolor}
\IfFileExists{xurl.sty}{\usepackage{xurl}}{} % add URL line breaks if available
\IfFileExists{bookmark.sty}{\usepackage{bookmark}}{\usepackage{hyperref}}
\hypersetup{
  pdftitle={Descriptive Statistics and Regression Tables in Word and LaTeX},
  hidelinks,
  pdfcreator={LaTeX via pandoc}}
\urlstyle{same} % disable monospaced font for URLs
\newif\ifbibliography
\usepackage{color}
\usepackage{fancyvrb}
\newcommand{\VerbBar}{|}
\newcommand{\VERB}{\Verb[commandchars=\\\{\}]}
\DefineVerbatimEnvironment{Highlighting}{Verbatim}{commandchars=\\\{\}}
% Add ',fontsize=\small' for more characters per line
\usepackage{framed}
\definecolor{shadecolor}{RGB}{248,248,248}
\newenvironment{Shaded}{\begin{snugshade}}{\end{snugshade}}
\newcommand{\AlertTok}[1]{\textcolor[rgb]{0.94,0.16,0.16}{#1}}
\newcommand{\AnnotationTok}[1]{\textcolor[rgb]{0.56,0.35,0.01}{\textbf{\textit{#1}}}}
\newcommand{\AttributeTok}[1]{\textcolor[rgb]{0.77,0.63,0.00}{#1}}
\newcommand{\BaseNTok}[1]{\textcolor[rgb]{0.00,0.00,0.81}{#1}}
\newcommand{\BuiltInTok}[1]{#1}
\newcommand{\CharTok}[1]{\textcolor[rgb]{0.31,0.60,0.02}{#1}}
\newcommand{\CommentTok}[1]{\textcolor[rgb]{0.56,0.35,0.01}{\textit{#1}}}
\newcommand{\CommentVarTok}[1]{\textcolor[rgb]{0.56,0.35,0.01}{\textbf{\textit{#1}}}}
\newcommand{\ConstantTok}[1]{\textcolor[rgb]{0.00,0.00,0.00}{#1}}
\newcommand{\ControlFlowTok}[1]{\textcolor[rgb]{0.13,0.29,0.53}{\textbf{#1}}}
\newcommand{\DataTypeTok}[1]{\textcolor[rgb]{0.13,0.29,0.53}{#1}}
\newcommand{\DecValTok}[1]{\textcolor[rgb]{0.00,0.00,0.81}{#1}}
\newcommand{\DocumentationTok}[1]{\textcolor[rgb]{0.56,0.35,0.01}{\textbf{\textit{#1}}}}
\newcommand{\ErrorTok}[1]{\textcolor[rgb]{0.64,0.00,0.00}{\textbf{#1}}}
\newcommand{\ExtensionTok}[1]{#1}
\newcommand{\FloatTok}[1]{\textcolor[rgb]{0.00,0.00,0.81}{#1}}
\newcommand{\FunctionTok}[1]{\textcolor[rgb]{0.00,0.00,0.00}{#1}}
\newcommand{\ImportTok}[1]{#1}
\newcommand{\InformationTok}[1]{\textcolor[rgb]{0.56,0.35,0.01}{\textbf{\textit{#1}}}}
\newcommand{\KeywordTok}[1]{\textcolor[rgb]{0.13,0.29,0.53}{\textbf{#1}}}
\newcommand{\NormalTok}[1]{#1}
\newcommand{\OperatorTok}[1]{\textcolor[rgb]{0.81,0.36,0.00}{\textbf{#1}}}
\newcommand{\OtherTok}[1]{\textcolor[rgb]{0.56,0.35,0.01}{#1}}
\newcommand{\PreprocessorTok}[1]{\textcolor[rgb]{0.56,0.35,0.01}{\textit{#1}}}
\newcommand{\RegionMarkerTok}[1]{#1}
\newcommand{\SpecialCharTok}[1]{\textcolor[rgb]{0.00,0.00,0.00}{#1}}
\newcommand{\SpecialStringTok}[1]{\textcolor[rgb]{0.31,0.60,0.02}{#1}}
\newcommand{\StringTok}[1]{\textcolor[rgb]{0.31,0.60,0.02}{#1}}
\newcommand{\VariableTok}[1]{\textcolor[rgb]{0.00,0.00,0.00}{#1}}
\newcommand{\VerbatimStringTok}[1]{\textcolor[rgb]{0.31,0.60,0.02}{#1}}
\newcommand{\WarningTok}[1]{\textcolor[rgb]{0.56,0.35,0.01}{\textbf{\textit{#1}}}}
\usepackage{graphicx,grffile}
\makeatletter
\def\maxwidth{\ifdim\Gin@nat@width>\linewidth\linewidth\else\Gin@nat@width\fi}
\def\maxheight{\ifdim\Gin@nat@height>\textheight\textheight\else\Gin@nat@height\fi}
\makeatother
% Scale images if necessary, so that they will not overflow the page
% margins by default, and it is still possible to overwrite the defaults
% using explicit options in \includegraphics[width, height, ...]{}
\setkeys{Gin}{width=\maxwidth,height=\maxheight,keepaspectratio}
% Set default figure placement to htbp
\makeatletter
\def\fps@figure{htbp}
\makeatother
\setlength{\emergencystretch}{3em} % prevent overfull lines
\providecommand{\tightlist}{%
  \setlength{\itemsep}{0pt}\setlength{\parskip}{0pt}}
\setcounter{secnumdepth}{-\maxdimen} % remove section numbering
\usepackage{booktabs}
\usepackage{longtable}
\usepackage{array}
\usepackage{multirow}
\usepackage{wrapfig}
\usepackage{float}
\usepackage{colortbl}
\usepackage{pdflscape}
\usepackage{tabu}
\usepackage{threeparttable}
\usepackage{threeparttablex}
\usepackage[normalem]{ulem}
\usepackage{makecell}
\usepackage{xcolor}

\title{Descriptive Statistics and Regression Tables in Word and LaTeX}
\author{Bas Machielsen\\
\href{mailto:a.h.machielsen@uu.nl}{\nolinkurl{a.h.machielsen@uu.nl}}\\
\href{https://github.com/basm92}{github.com/basm92}}
\date{\today}
\institute{Utrecht University - R Cafe}

\begin{document}
\frame{\titlepage}

\begin{frame}[fragile]{Possibilities and packages}
\protect\hypertarget{possibilities-and-packages}{}

A small disclaimer..

\begin{itemize}
\tightlist
\item
  There are various possibilities and packages in R to make tidy,
  well-formatted tables
\item
  I do not claim to know, let alone be familiar with, even a tiny
  fraction of them
\item
  I do know how to use a few particular ones, most notable
  \texttt{sjPlot} and \texttt{stargazer}, which are among the most
  popular and most-used packages for this purpose, whose features I will
  demonstrate throughout this lecture
\item
  I will also show a couple of alternatives to circumvent the
  shortcomings of these packages.
\item
  Stargazer has a vignette available
  \href{https://cran.r-project.org/web/packages/stargazer/vignettes/stargazer.pdf}{\textbf{here}},
  which is an excellent tutorial in its own right, and sjPlot has a few
  available tutorials
  \href{http://www.strengejacke.de/sjPlot/articles/}{\textbf{here}}
\end{itemize}

\end{frame}

\begin{frame}{Why use them?}
\protect\hypertarget{why-use-them}{}

\begin{itemize}
\tightlist
\item
  What is stargazer?
\end{itemize}

\begin{quote}
stargazer is an R package that creates LATEX code, HTML code and ASCII
text for well-formatted regression tables, with multiple models
side-by-side, as well as for summary statistics tables, data frames,
vectors and matrices.
\end{quote}

\begin{itemize}
\tightlist
\item
  Why should you use stargazer? From the vignette:
\end{itemize}

\begin{quote}
Compared to available alternatives, stargazer excels in at least three
respects: its ease of use, the large number of models it supports, and
its beautiful aesthetics. These advantages have made it the R-to-LATEX
package of choice for many satisfied users at research and teaching
institutions around the world.
\end{quote}

\end{frame}

\begin{frame}{Why use them?}
\protect\hypertarget{why-use-them-1}{}

\begin{itemize}
\tightlist
\item
  What is sjPlot?
\end{itemize}

\begin{quote}
{[}a{]} collection of plotting and table output functions for data
visualization. Results of various statistical analyses (that are
commonly used in social sciences) can be visualized using this package,
including simple and cross tabulated frequencies, histograms, box plots,
(generalized) linear models, mixed effects models, PCA and correlation
matrices, cluster analyses, scatter plots, Likert scales, effects plots
of interaction terms in regression models, constructing index or score
variables and much more.
\end{quote}

\begin{itemize}
\tightlist
\item
  sjPlot has way more features than stargazer. We will not focus on
  these features in this session.
\end{itemize}

\end{frame}

\begin{frame}[fragile]{Descriptive statistics - stargazer}
\protect\hypertarget{descriptive-statistics---stargazer}{}

We will start out by using stargazer in \LaTeX format. The advantage of
a \LaTeX format is that you can use it right away in a presentation,
such as this one!

\begin{Shaded}
\begin{Highlighting}[]
\CommentTok{#Load an example dataset}
\CommentTok{#the data comes from Hamermesh and Parker (2005)}
\CommentTok{#about impact of beauty on teacher evaluations}

\KeywordTok{data}\NormalTok{(TeachingRatings)}

\KeywordTok{stargazer}\NormalTok{(TeachingRatings, }
          \DataTypeTok{header=}\OtherTok{FALSE}\NormalTok{, }\DataTypeTok{type=}\StringTok{'latex'}\NormalTok{,}
          \DataTypeTok{summary.stat =} \KeywordTok{c}\NormalTok{(}\StringTok{"mean"}\NormalTok{,}\StringTok{"min"}\NormalTok{,}\StringTok{"max"}\NormalTok{,}\StringTok{"n"}\NormalTok{), }
          \DataTypeTok{font.size =} \StringTok{"footnotesize"}\NormalTok{)}
\end{Highlighting}
\end{Shaded}

\end{frame}

\begin{frame}{Descriptive statistics - stargazer}
\protect\hypertarget{descriptive-statistics---stargazer-1}{}

\begin{itemize}
\tightlist
\item
  This is the output you get:
\end{itemize}

\begin{table}[!htbp] \centering 
  \caption{} 
  \label{} 
\footnotesize 
\begin{tabular}{@{\extracolsep{5pt}}lcccc} 
\\[-1.8ex]\hline 
\hline \\[-1.8ex] 
Statistic & \multicolumn{1}{c}{N} & \multicolumn{1}{c}{Mean} & \multicolumn{1}{c}{Min} & \multicolumn{1}{c}{Max} \\ 
\hline \\[-1.8ex] 
age & 463 & 48.365 & 29 & 73 \\ 
beauty & 463 & 0.00000 & $-$1.450 & 1.970 \\ 
eval & 463 & 3.998 & 2.100 & 5.000 \\ 
students & 463 & 36.624 & 5 & 380 \\ 
allstudents & 463 & 55.177 & 8 & 581 \\ 
\hline \\[-1.8ex] 
\end{tabular} 
\end{table}

\end{frame}

\begin{frame}[fragile]{Descriptive statistics - stargazer}
\protect\hypertarget{descriptive-statistics---stargazer-2}{}

\begin{itemize}
\tightlist
\item
  The option list of stargazer is very extensive. For example, the same
  output can be achieved by omitting several (non-default) statistics:
\end{itemize}

\begin{Shaded}
\begin{Highlighting}[]
\CommentTok{#Load an example dataset}
\CommentTok{#the data comes from a Hamermesh and Parker (2005)}
\CommentTok{#about impact of beauty on teacher evaluations}

\KeywordTok{data}\NormalTok{(TeachingRatings)}

\KeywordTok{stargazer}\NormalTok{(TeachingRatings, }
          \DataTypeTok{header=}\OtherTok{FALSE}\NormalTok{, }\DataTypeTok{type=}\StringTok{'latex'}\NormalTok{,}
          \DataTypeTok{omit.summary.stat =} \KeywordTok{c}\NormalTok{(}\StringTok{"sd"}\NormalTok{, }\StringTok{"p25"}\NormalTok{, }\StringTok{"p75"}\NormalTok{), }
          \DataTypeTok{font.size =} \StringTok{"footnotesize"}\NormalTok{,}
          \DataTypeTok{title =} \StringTok{"Hello, R Cafe!"}
\NormalTok{          )}
\end{Highlighting}
\end{Shaded}

\end{frame}

\begin{frame}{Descriptive statistics - stargazer}
\protect\hypertarget{descriptive-statistics---stargazer-3}{}

The output you get is the following:

\begin{table}[!htbp] \centering 
  \caption{Hello, R Cafe!} 
  \label{} 
\footnotesize 
\begin{tabular}{@{\extracolsep{5pt}}lcccc} 
\\[-1.8ex]\hline 
\hline \\[-1.8ex] 
Statistic & \multicolumn{1}{c}{N} & \multicolumn{1}{c}{Mean} & \multicolumn{1}{c}{Min} & \multicolumn{1}{c}{Max} \\ 
\hline \\[-1.8ex] 
age & 463 & 48.365 & 29 & 73 \\ 
beauty & 463 & 0.00000 & $-$1.450 & 1.970 \\ 
eval & 463 & 3.998 & 2.100 & 5.000 \\ 
students & 463 & 36.624 & 5 & 380 \\ 
allstudents & 463 & 55.177 & 8 & 581 \\ 
\hline \\[-1.8ex] 
\end{tabular} 
\end{table}

Please refer to the stargazer vignette
\href{https://cran.r-project.org/web/packages/stargazer/stargazer.pdf\#stargazer_summary_stat_code_list}{\textbf{here}}
to look for the specific summary statistic codes.

\end{frame}

\begin{frame}[fragile]{Descriptive statistics - stargazer}
\protect\hypertarget{descriptive-statistics---stargazer-4}{}

\begin{itemize}
\tightlist
\item
  Hopefully the advantages have been clear. The syntax is amazingly
  simple and the number of options is huge (Have a look at
  \texttt{?stargazer}). It automatically filters NA observations. It is
  also easy to integrate these tables into (R)Markdown and
  \LaTeX documents.
\end{itemize}

Now, a few disadvantages..

\begin{itemize}
\item
  You have to specify the option \texttt{header\ =\ FALSE} in order to
  bypass the default output which contains the credit of the package
  creator (try it!)
\item
  stargazer automatically extracts the variables which are suitable for
  descriptive statistics out of your dataset. If you want a summarise
  of, e.g., factor variables, you have to convert them.
\end{itemize}

\end{frame}

\begin{frame}[fragile]{Descriptive statistics - stargazer in Word}
\protect\hypertarget{descriptive-statistics---stargazer-in-word}{}

\begin{itemize}
\item
  Stargazer also supports exporting tables to Microsoft Word.
\item
  It is possible to export tables in one file, and to overwrite or
  append these documents. (This is also supported for .tex files.)
\end{itemize}

\footnotesize

\begin{Shaded}
\begin{Highlighting}[]
\CommentTok{#Let's try to extract another dataset, CPS1988,}
\CommentTok{#about the determinants of wages}
\CommentTok{#and summarise the descriptives in a Word table.}
\KeywordTok{data}\NormalTok{(CPS1988)}

\KeywordTok{stargazer}\NormalTok{(CPS1988[CPS1988}\OperatorTok{$}\NormalTok{ethnicity }\OperatorTok{==}\StringTok{ "cauc"}\NormalTok{,],}
          \DataTypeTok{header=}\OtherTok{FALSE}\NormalTok{, }\DataTypeTok{type=}\StringTok{'latex'}\NormalTok{,}
          \DataTypeTok{omit.summary.stat =} \KeywordTok{c}\NormalTok{(}\StringTok{"p25"}\NormalTok{, }\StringTok{"p75"}\NormalTok{), }
          \DataTypeTok{font.size =} \StringTok{"footnotesize"}\NormalTok{,}
          \DataTypeTok{title =} \StringTok{"Caucasian"}\NormalTok{)}

\KeywordTok{stargazer}\NormalTok{(CPS1988[CPS1988}\OperatorTok{$}\NormalTok{ethnicity }\OperatorTok{==}\StringTok{ "afam"}\NormalTok{,],}
          \DataTypeTok{header=}\OtherTok{FALSE}\NormalTok{, }\DataTypeTok{type=}\StringTok{'latex'}\NormalTok{,}
          \DataTypeTok{omit.summary.stat =} \KeywordTok{c}\NormalTok{(}\StringTok{"p25"}\NormalTok{, }\StringTok{"p75"}\NormalTok{), }
          \DataTypeTok{font.size =} \StringTok{"footnotesize"}\NormalTok{,}
          \DataTypeTok{title =} \StringTok{"Caucasian"}\NormalTok{)}
\end{Highlighting}
\end{Shaded}

\normalsize

\end{frame}

\begin{frame}{Descriptive statistics - stargazer in Word}
\protect\hypertarget{descriptive-statistics---stargazer-in-word-1}{}

This is the output of the two tables:

\begin{table}[!htbp] \centering 
  \caption{Caucasian} 
  \label{} 
\footnotesize 
\begin{tabular}{@{\extracolsep{5pt}}lccccc} 
\\[-1.8ex]\hline 
\hline \\[-1.8ex] 
Statistic & \multicolumn{1}{c}{N} & \multicolumn{1}{c}{Mean} & \multicolumn{1}{c}{St. Dev.} & \multicolumn{1}{c}{Min} & \multicolumn{1}{c}{Max} \\ 
\hline \\[-1.8ex] 
wage & 25,923 & 617.234 & 461.210 & 50.050 & 18,777.200 \\ 
education & 25,923 & 13.132 & 2.902 & 0 & 18 \\ 
experience & 25,923 & 18.153 & 13.040 & $-$4 & 63 \\ 
\hline \\[-1.8ex] 
\end{tabular} 
\end{table}

\begin{table}[!htbp] \centering 
  \caption{African American} 
  \label{} 
\footnotesize 
\begin{tabular}{@{\extracolsep{5pt}}lccccc} 
\\[-1.8ex]\hline 
\hline \\[-1.8ex] 
Statistic & \multicolumn{1}{c}{N} & \multicolumn{1}{c}{Mean} & \multicolumn{1}{c}{St. Dev.} & \multicolumn{1}{c}{Min} & \multicolumn{1}{c}{Max} \\ 
\hline \\[-1.8ex] 
wage & 2,232 & 446.853 & 312.437 & 52.230 & 3,527.340 \\ 
education & 2,232 & 12.327 & 2.767 & 0 & 18 \\ 
experience & 2,232 & 18.741 & 13.513 & $-$2 & 61 \\ 
\hline \\[-1.8ex] 
\end{tabular} 
\end{table}

\end{frame}

\begin{frame}[fragile]{Descriptive statistics - stargazer in Word}
\protect\hypertarget{descriptive-statistics---stargazer-in-word-2}{}

\begin{itemize}
\item
  Exporting tables to Word is possible ``natively'' in 2 ways:

  \begin{enumerate}
  \tightlist
  \item
    First, output to .html, and copy the resulting tables from your web
    browser to Word
  \item
    Second, output each separate stargazer command to different .doc
    documents
  \end{enumerate}
\item
  Stargazer does not yet support appending existing documents with new
  tables, and automatically overwrites tables.
\end{itemize}

Syntax:

\footnotesize

\begin{Shaded}
\begin{Highlighting}[]
\KeywordTok{stargazer}\NormalTok{(CPS1988[CPS1988}\OperatorTok{$}\NormalTok{ethnicity }\OperatorTok{==}\StringTok{ "cauc"}\NormalTok{,],}
            \DataTypeTok{header=}\OtherTok{FALSE}\NormalTok{, }\DataTypeTok{omit.summary.stat =} \KeywordTok{c}\NormalTok{(}\StringTok{"p25"}\NormalTok{, }\StringTok{"p75"}\NormalTok{),}
            \DataTypeTok{font.size =} \StringTok{"footnotesize"}\NormalTok{,}
            \DataTypeTok{title =} \StringTok{"Caucasian"}\NormalTok{,}
            \DataTypeTok{type=}\StringTok{'html'}\NormalTok{, }\DataTypeTok{out =} \StringTok{"test1.doc"}\NormalTok{) }\CommentTok{#These 2 arguments are the only 2 you }
                                            \CommentTok{#need}

\KeywordTok{stargazer}\NormalTok{(CPS1988[CPS1988}\OperatorTok{$}\NormalTok{ethnicity }\OperatorTok{==}\StringTok{ "afam"}\NormalTok{,],}
          \DataTypeTok{header=}\OtherTok{FALSE}\NormalTok{, }\DataTypeTok{omit.summary.stat =} \KeywordTok{c}\NormalTok{(}\StringTok{"p25"}\NormalTok{, }\StringTok{"p75"}\NormalTok{), }
          \DataTypeTok{font.size =} \StringTok{"footnotesize"}\NormalTok{,}
          \DataTypeTok{title =} \StringTok{"African American"}\NormalTok{,}
          \DataTypeTok{type=}\StringTok{'html'}\NormalTok{, }\DataTypeTok{out =} \StringTok{"test2.doc"}\NormalTok{) }\CommentTok{#These 2 arguments are the only 2 you }
                                          \CommentTok{#need}
\end{Highlighting}
\end{Shaded}

\normalsize

\end{frame}

\begin{frame}[fragile]{Descriptive statistics - stargazer in Word}
\protect\hypertarget{descriptive-statistics---stargazer-in-word-3}{}

\begin{itemize}
\tightlist
\item
  However, if you want to append, you can use a little bit of R's power
  in the following way:
\end{itemize}

\footnotesize

\begin{Shaded}
\begin{Highlighting}[]
\CommentTok{#Step 1, make a list for all the }
\CommentTok{#partitions of the data, include names}
\NormalTok{models <-}\StringTok{  }\KeywordTok{list}\NormalTok{(}
    \DataTypeTok{Caucasian =}\NormalTok{ CPS1988[CPS1988}\OperatorTok{$}\NormalTok{ethnicity }\OperatorTok{==}\StringTok{ "cauc"}\NormalTok{,],}
    \DataTypeTok{AfroAmerican =}\NormalTok{ CPS1988[CPS1988}\OperatorTok{$}\NormalTok{ethnicity }\OperatorTok{==}\StringTok{ "afam"}\NormalTok{,],}
    \DataTypeTok{Northeast =}\NormalTok{ CPS1988[CPS1988}\OperatorTok{$}\NormalTok{region }\OperatorTok{==}\StringTok{ "northeast"}\NormalTok{,],}
    \DataTypeTok{Midwest =}\NormalTok{ CPS1988[CPS1988}\OperatorTok{$}\NormalTok{region }\OperatorTok{==}\StringTok{ "midwest"}\NormalTok{,],}
    \DataTypeTok{South =}\NormalTok{ CPS1988[CPS1988}\OperatorTok{$}\NormalTok{region }\OperatorTok{==}\StringTok{ "south"}\NormalTok{,],}
    \DataTypeTok{West =}\NormalTok{ CPS1988[CPS1988}\OperatorTok{$}\NormalTok{region }\OperatorTok{==}\StringTok{ "west"}\NormalTok{,]}
\NormalTok{)}

\CommentTok{#Step 2, for loop for every element in the list you want to print}
\ControlFlowTok{for}\NormalTok{ (m }\ControlFlowTok{in} \DecValTok{1}\OperatorTok{:}\KeywordTok{length}\NormalTok{(models))\{}
\NormalTok{  a =}\StringTok{ }\KeywordTok{vector}\NormalTok{(}\DataTypeTok{length =} \KeywordTok{length}\NormalTok{(models))}
\NormalTok{  a[m] =}\StringTok{ }\NormalTok{(}\KeywordTok{names}\NormalTok{(models[m]))}
\NormalTok{  s =}\StringTok{ }\KeywordTok{capture.output}\NormalTok{(}\KeywordTok{stargazer}\NormalTok{(models[m], }\DataTypeTok{type =} \StringTok{"html"}\NormalTok{, }\DataTypeTok{title =}\NormalTok{ a[m]))}
  \KeywordTok{cat}\NormalTok{(}\KeywordTok{paste}\NormalTok{(s,}\StringTok{"}\CharTok{\textbackslash{}n}\StringTok{"}\NormalTok{),}\DataTypeTok{file=}\StringTok{"tables.doc"}\NormalTok{,}\DataTypeTok{append=}\OtherTok{TRUE}\NormalTok{)}
  \KeywordTok{cat}\NormalTok{(}\StringTok{" "}\NormalTok{,}\DataTypeTok{file=}\StringTok{"tables.doc"}\NormalTok{,}\DataTypeTok{append=}\OtherTok{TRUE}\NormalTok{)}
\NormalTok{\}}
\end{Highlighting}
\end{Shaded}

\normalsize

\end{frame}

\begin{frame}{Descriptive statistics - stargazer in Word}
\protect\hypertarget{descriptive-statistics---stargazer-in-word-4}{}

\begin{itemize}
\item
  One only has to put all the partitions in a list, change the filename,
  and change the desired descriptive statistics accordingly.
\item
  Let us now look at the \href{tables.doc}{\textbf{document in which all
  tables are}}! This is very useful when organizing all tables and/or
  graphs for a paper.
\end{itemize}

\end{frame}

\begin{frame}[fragile]{Descriptives in sjPlot}
\protect\hypertarget{descriptives-in-sjplot}{}

\begin{itemize}
\tightlist
\item
  It is also possible to use \texttt{sjPlot} for descriptive statistics.
  \footnotesize
\end{itemize}

\begin{Shaded}
\begin{Highlighting}[]
\CommentTok{#Let's use a dataset of determinants of GDP Growth}
\KeywordTok{data}\NormalTok{(GrowthDJ)}
\NormalTok{descriptives <-}\StringTok{ }\KeywordTok{descr}\NormalTok{(GrowthDJ, }\DataTypeTok{show =} 
                        \KeywordTok{c}\NormalTok{(}\StringTok{"type"}\NormalTok{,}\StringTok{"label"}\NormalTok{,}\StringTok{"n"}\NormalTok{,}\StringTok{"mean"}\NormalTok{,}\StringTok{"sd"}\NormalTok{))}

\KeywordTok{kable}\NormalTok{(descriptives, }\DataTypeTok{caption =} \StringTok{"Descriptive Statistics"}\NormalTok{, }
      \DataTypeTok{booktabs =} \OtherTok{TRUE}\NormalTok{, }\DataTypeTok{row.names =} \OtherTok{FALSE}\NormalTok{) }\OperatorTok
\StringTok{  }\KeywordTok{kable_styling}\NormalTok{(}\DataTypeTok{latex_options =} \StringTok{"striped"}\NormalTok{)}
\end{Highlighting}
\end{Shaded}

\normalsize

\end{frame}

\begin{frame}{Descriptives in sjPlot}
\protect\hypertarget{descriptives-in-sjplot-1}{}

\begin{table}[t]

\caption{\label{tab:unnamed-chunk-11}Descriptive Statistics}
\centering
\begin{tabular}{lllrrr}
\toprule
var & type & label & n & mean & sd\\
\midrule
\rowcolor{gray!6}  oil & categorical & oil & 121 & 1.809917 & 0.3939977\\
inter & categorical & inter & 121 & 1.619835 & 0.4874457\\
\rowcolor{gray!6}  oecd & categorical & oecd & 121 & 1.181818 & 0.3872983\\
gdp60 & numeric & gdp60 & 116 & 3681.818966 & 7492.8776368\\
\rowcolor{gray!6}  gdp85 & numeric & gdp85 & 108 & 5683.259259 & 5688.6708192\\
\addlinespace
gdpgrowth & numeric & gdpgrowth & 117 & 4.094017 & 1.8914641\\
\rowcolor{gray!6}  popgrowth & numeric & popgrowth & 107 & 2.279439 & 0.9987481\\
invest & numeric & invest & 121 & 18.157025 & 7.8533096\\
\rowcolor{gray!6}  school & numeric & school & 118 & 5.526271 & 3.5320372\\
literacy60 & numeric & literacy60 & 103 & 48.165048 & 35.3542568\\
\bottomrule
\end{tabular}
\end{table}

\end{frame}

\begin{frame}[fragile]{Descriptives in sjPlot}
\protect\hypertarget{descriptives-in-sjplot-2}{}

\begin{itemize}
\tightlist
\item
  \texttt{sjPlot} has a lot of advantages. It automatically omits NA
  observations, and transforms factor variables to numeric variables
  (although this is dangerous!).
\item
  It can also make contingency tables, which stargazer cannot (readily)
  do:
\end{itemize}

\begin{Shaded}
\begin{Highlighting}[]
\KeywordTok{sjt.xtab}\NormalTok{(GrowthDJ}\OperatorTok{$}\NormalTok{oil, GrowthDJ}\OperatorTok{$}\NormalTok{oecd, }
         \DataTypeTok{use.viewer =} \OtherTok{TRUE}\NormalTok{, }\DataTypeTok{file =} \StringTok{"contingency.doc"}\NormalTok{)}
\end{Highlighting}
\end{Shaded}

\begin{figure}
\centering
\includegraphics{contingency.png}
\caption{Figure}
\end{figure}

\end{frame}

\begin{frame}[fragile]{Correlation tables}
\protect\hypertarget{correlation-tables}{}

\begin{itemize}
\tightlist
\item
  Both \texttt{sjPlot} and \texttt{stargazer} can make correlation
  tables. First, let's try stargazer:
\end{itemize}

\begin{Shaded}
\begin{Highlighting}[]
\KeywordTok{data}\NormalTok{(GrowthDJ)}
\NormalTok{cormat <-}\StringTok{ }\KeywordTok{select_if}\NormalTok{(GrowthDJ, is.numeric)}
\end{Highlighting}
\end{Shaded}

\begin{Shaded}
\begin{Highlighting}[]
\KeywordTok{stargazer}\NormalTok{(}\KeywordTok{cor}\NormalTok{(cormat[}\DecValTok{5}\OperatorTok{:}\DecValTok{7}\NormalTok{], }
              \DataTypeTok{use =} \KeywordTok{c}\NormalTok{(}\StringTok{'pairwise.complete.obs'}\NormalTok{)),}
          \DataTypeTok{header=}\OtherTok{FALSE}\NormalTok{, }\DataTypeTok{type=}\StringTok{'latex'}\NormalTok{,}
          \DataTypeTok{omit.summary.stat =} \KeywordTok{c}\NormalTok{(}\StringTok{"p25"}\NormalTok{, }\StringTok{"p75"}\NormalTok{), }
          \DataTypeTok{font.size =} \StringTok{"footnotesize"}\NormalTok{,}
          \DataTypeTok{title =} \StringTok{"Correlation Matrix"}\NormalTok{)}
\end{Highlighting}
\end{Shaded}

\begin{itemize}
\tightlist
\item
  Stargazer has no native stars, a feature which sjPlot does have.
\end{itemize}

\end{frame}

\begin{frame}[fragile]{Correlation tables - stargazer}
\protect\hypertarget{correlation-tables---stargazer}{}

\begin{itemize}
\tightlist
\item
  This is the output:
\end{itemize}

\begin{table}[!htbp] \centering 
  \caption{Correlation Matrix} 
  \label{} 
\footnotesize 
\begin{tabular}{@{\extracolsep{5pt}} cccc} 
\\[-1.8ex]\hline 
\hline \\[-1.8ex] 
 & invest & school & literacy60 \\ 
\hline \\[-1.8ex] 
invest & $1$ & $0.622$ & $0.639$ \\ 
school & $0.622$ & $1$ & $0.818$ \\ 
literacy60 & $0.639$ & $0.818$ & $1$ \\ 
\hline \\[-1.8ex] 
\end{tabular} 
\end{table}

\begin{itemize}
\tightlist
\item
  \LaTeX-users: \texttt{stargazer} cannot correctly handle negative
  numbers when generating the \LaTeX code, causing some compilers to
  have difficulties.
\item
  Word and \LaTeX-users: you can also use stargazer's
  \texttt{out}-option to export the tables to Word and .tex.
\end{itemize}

\end{frame}

\begin{frame}[fragile]{Correlation tables}
\protect\hypertarget{correlation-tables-1}{}

\begin{itemize}
\tightlist
\item
  I can also use the package \texttt{xtable}, which surpasses the
  problem of stargazer but creates otherwise identical tables (see
  .Rmd-file for the code).
\end{itemize}

\begin{table}[ht]
\centering
\begin{tabular}{rrrrrrr}
  \hline
 & gdp60 & gdp85 & gdpgrowth & popgrowth & invest & school \\ 
  \hline
gdp60 & 1.000 & 0.631 & -0.122 & 0.291 & 0.091 & 0.337 \\ 
  gdp85 & 0.631 & 1.000 & 0.139 & -0.222 & 0.581 & 0.697 \\ 
  gdpgrowth & -0.122 & 0.139 & 1.000 & 0.242 & 0.351 & 0.198 \\ 
  popgrowth & 0.291 & -0.222 & 0.242 & 1.000 & -0.332 & -0.213 \\ 
  invest & 0.091 & 0.581 & 0.351 & -0.332 & 1.000 & 0.622 \\ 
  school & 0.337 & 0.697 & 0.198 & -0.213 & 0.622 & 1.000 \\ 
   \hline
\end{tabular}
\caption{Correlation Matrix} 
\end{table}

\begin{itemize}
\tightlist
\item
  \texttt{xtable} also supports output in \LaTeX or html format.
\end{itemize}

\end{frame}

\begin{frame}[fragile]{Correlation tables - sjPlot}
\protect\hypertarget{correlation-tables---sjplot}{}

\begin{itemize}
\tightlist
\item
  \texttt{sjPlot} has a range of functions to visualize correlations.
  For example:
\end{itemize}

\begin{Shaded}
\begin{Highlighting}[]
\KeywordTok{sjt.corr}\NormalTok{(cormat[}\DecValTok{1}\OperatorTok{:}\DecValTok{6}\NormalTok{], }\DataTypeTok{file =} \StringTok{"cortab.doc"}\NormalTok{)}
\end{Highlighting}
\end{Shaded}

\begin{figure}
\centering
\includegraphics{cortab.png}
\caption{Correlation table}
\end{figure}

\end{frame}

\begin{frame}[fragile]{Correlation tables - sjPlot}
\protect\hypertarget{correlation-tables---sjplot-1}{}

\begin{itemize}
\tightlist
\item
  Another example (this uses \texttt{ggplot2}) \footnotesize
\end{itemize}

\begin{Shaded}
\begin{Highlighting}[]
\KeywordTok{sjp.corr}\NormalTok{(cormat[}\DecValTok{1}\OperatorTok{:}\DecValTok{5}\NormalTok{], }\DataTypeTok{title =} \StringTok{"Correlation Matrix"}\NormalTok{)}
\end{Highlighting}
\end{Shaded}

\includegraphics{PresentationRCafe_files/figure-beamer/unnamed-chunk-18-1.pdf}
\normalsize

\end{frame}

\begin{frame}[fragile]{Correlation tables - with stars}
\protect\hypertarget{correlation-tables---with-stars}{}

\begin{itemize}
\item
  If you would like a correlation function that prints correlation
  tables with stars indicating significance, have a look
  \href{http://myowelt.blogspot.com/2008/04/beautiful-correlation-tables-in-r.html}{\textbf{here}}
\item
  Example:
\end{itemize}

\begin{Shaded}
\begin{Highlighting}[]
\KeywordTok{xtable}\NormalTok{(}\KeywordTok{corstarsl}\NormalTok{(cormat[,}\DecValTok{1}\OperatorTok{:}\DecValTok{6}\NormalTok{]), }
       \DataTypeTok{caption =} \StringTok{"Correlation Table with Stars"}\NormalTok{)}
\end{Highlighting}
\end{Shaded}

\begin{table}[ht]
\centering
\begin{tabular}{rlllll}
  \hline
 & gdp60 & gdp85 & gdpgrowth & popgrowth & invest \\ 
  \hline
gdp60 &  &  &  &  &  \\ 
  gdp85 &  0.63*** &  &  &  &  \\ 
  gdpgrowth & -0.12  &  0.14  &  &  &  \\ 
  popgrowth &  0.29**  & -0.22*  &  0.24*  &  &  \\ 
  invest &  0.09  &  0.58*** &  0.35*** & -0.33*** &  \\ 
  school &  0.34*** &  0.70*** &  0.20*  & -0.21*  &  0.62*** \\ 
   \hline
\end{tabular}
\caption{Correlation Table with Stars} 
\end{table}

\end{frame}

\begin{frame}[fragile]{Regression tables}
\protect\hypertarget{regression-tables}{}

\begin{itemize}
\item
  Now, finally, we can use stargazer's principal application, regression
  tables.
\item
  Everything is \emph{very} straightforward: just look at this:
  \footnotesize
\end{itemize}

\begin{Shaded}
\begin{Highlighting}[]
\KeywordTok{data}\NormalTok{(TeachingRatings) }
\NormalTok{model1 <-}\StringTok{ }\KeywordTok{lm}\NormalTok{(}\DataTypeTok{data =}\NormalTok{ TeachingRatings, }
\NormalTok{             eval }\OperatorTok{~}\StringTok{ }\NormalTok{beauty)}
\NormalTok{model2 <-}\StringTok{ }\KeywordTok{lm}\NormalTok{(}\DataTypeTok{data =}\NormalTok{TeachingRatings, }
\NormalTok{             eval }\OperatorTok{~}\StringTok{ }\NormalTok{beauty }\OperatorTok{+}\StringTok{ }\NormalTok{age)}
\NormalTok{model3 <-}\StringTok{ }\KeywordTok{lm}\NormalTok{(}\DataTypeTok{data =}\NormalTok{TeachingRatings, }
\NormalTok{             eval }\OperatorTok{~}\StringTok{ }\NormalTok{beauty }\OperatorTok{+}\StringTok{ }\NormalTok{age }\OperatorTok{+}\StringTok{ }\NormalTok{gender)}
\NormalTok{model4 <-}\StringTok{ }\KeywordTok{lm}\NormalTok{(}\DataTypeTok{data =}\NormalTok{TeachingRatings, }
\NormalTok{             eval }\OperatorTok{~}\StringTok{ }\NormalTok{beauty }\OperatorTok{+}\StringTok{ }\NormalTok{age }\OperatorTok{+}\StringTok{ }\NormalTok{gender }\OperatorTok{+}\StringTok{ }\NormalTok{students)}

\KeywordTok{stargazer}\NormalTok{(model1, model2, model3, model4,}
          \DataTypeTok{header =} \OtherTok{FALSE}\NormalTok{,}
          \DataTypeTok{caption =} \OtherTok{FALSE}\NormalTok{,}
          \DataTypeTok{font.size =} \StringTok{"footnotesize"}\NormalTok{,}
          \DataTypeTok{column.sep.width =} \StringTok{"0pt"}\NormalTok{,}
          \DataTypeTok{omit.stat =} \KeywordTok{c}\NormalTok{(}\StringTok{"ll"}\NormalTok{, }\StringTok{"F"}\NormalTok{,}\StringTok{"ser"}\NormalTok{))}
\end{Highlighting}
\end{Shaded}

\normalsize

\end{frame}

\begin{frame}{Regression tables}
\protect\hypertarget{regression-tables-1}{}

\def\arraystretch{0.7}

\begin{table}[!htbp] \centering 
  \caption{} 
  \label{} 
\footnotesize 
\begin{tabular}{@{\extracolsep{0pt}}lcccc} 
\\[-1.8ex]\hline 
\hline \\[-1.8ex] 
 & \multicolumn{4}{c}{\textit{Dependent variable:}} \\ 
\cline{2-5} 
\\[-1.8ex] & \multicolumn{4}{c}{eval} \\ 
\\[-1.8ex] & (1) & (2) & (3) & (4)\\ 
\hline \\[-1.8ex] 
 beauty & 0.133$^{***}$ & 0.134$^{***}$ & 0.140$^{***}$ & 0.141$^{***}$ \\ 
  & (0.032) & (0.034) & (0.033) & (0.034) \\ 
  & & & & \\ 
 age &  & 0.0003 & $-$0.003 & $-$0.003 \\ 
  &  & (0.003) & (0.003) & (0.003) \\ 
  & & & & \\ 
 genderfemale &  &  & $-$0.211$^{***}$ & $-$0.212$^{***}$ \\ 
  &  &  & (0.053) & (0.053) \\ 
  & & & & \\ 
 students &  &  &  & $-$0.0001 \\ 
  &  &  &  & (0.001) \\ 
  & & & & \\ 
 Constant & 3.998$^{***}$ & 3.984$^{***}$ & 4.213$^{***}$ & 4.220$^{***}$ \\ 
  & (0.025) & (0.134) & (0.144) & (0.146) \\ 
  & & & & \\ 
\hline \\[-1.8ex] 
Observations & 463 & 463 & 463 & 463 \\ 
R$^{2}$ & 0.036 & 0.036 & 0.068 & 0.068 \\ 
Adjusted R$^{2}$ & 0.034 & 0.032 & 0.062 & 0.060 \\ 
\hline 
\hline \\[-1.8ex] 
\textit{Note:}  & \multicolumn{4}{r}{$^{*}$p$<$0.1; $^{**}$p$<$0.05; $^{***}$p$<$0.01} \\ 
\end{tabular} 
\end{table}

\begin{table}[!htbp] \centering 
  \caption{} 
  \label{} 
\footnotesize 
\begin{tabular}{@{\extracolsep{0pt}} c} 
\\[-1.8ex]\hline 
\hline \\[-1.8ex] 
FALSE \\ 
\hline \\[-1.8ex] 
\end{tabular} 
\end{table}

\end{frame}

\begin{frame}[fragile]{Regression tables}
\protect\hypertarget{regression-tables-2}{}

\begin{itemize}
\tightlist
\item
  \texttt{stargazer} can also change styles to fit standard formats
  required by journals in the social sciences, and automatically
  incorporates different dependent variables.
\end{itemize}

\def\arraystretch{0.7}

\begin{table}[!htbp] \centering 
  \caption{} 
  \label{} 
\scriptsize 
\begin{tabular}{@{\extracolsep{0pt}}lcccc} 
\\[-1.8ex]\hline 
\hline \\[-1.8ex] 
 & \multicolumn{4}{c}{\textit{Dependent variable:}} \\ 
\cline{2-5} 
\\[-1.8ex] & \multicolumn{2}{c}{eval} & \multicolumn{2}{c}{beauty} \\ 
\\[-1.8ex] & (1) & (2) & (3) & (4)\\ 
\hline \\[-1.8ex] 
 beauty & 0.133$^{***}$ & 0.134$^{***}$ &  &  \\ 
  & (0.032) & (0.034) &  &  \\ 
  & & & & \\ 
 age &  & 0.0003 &  & $-$0.024$^{***}$ \\ 
  &  & (0.003) &  & (0.004) \\ 
  & & & & \\ 
 eval &  &  & 0.270$^{***}$ &  \\ 
  &  &  & (0.066) &  \\ 
  & & & & \\ 
 tenureyes &  &  & 0.010 &  \\ 
  &  &  & (0.088) &  \\ 
  & & & & \\ 
 Constant & 3.998$^{***}$ & 3.984$^{***}$ & $-$1.086$^{***}$ & 1.159$^{***}$ \\ 
  & (0.025) & (0.134) & (0.282) & (0.177) \\ 
  & & & & \\ 
\hline \\[-1.8ex] 
Observations & 463 & 463 & 463 & 463 \\ 
R$^{2}$ & 0.036 & 0.036 & 0.036 & 0.089 \\ 
Adjusted R$^{2}$ & 0.034 & 0.032 & 0.032 & 0.087 \\ 
\hline 
\hline \\[-1.8ex] 
\textit{Note:}  & \multicolumn{4}{r}{$^{*}$p$<$0.1; $^{**}$p$<$0.05; $^{***}$p$<$0.01} \\ 
\end{tabular} 
\end{table}

\begin{table}[!htbp] \centering 
  \caption{} 
  \label{} 
\scriptsize 
\begin{tabular}{@{\extracolsep{0pt}} c} 
\\[-1.8ex]\hline 
\hline \\[-1.8ex] 
 \\ 
\hline \\[-1.8ex] 
\end{tabular} 
\end{table}

\end{frame}

\begin{frame}[fragile]{Standard errors}
\protect\hypertarget{standard-errors}{}

The issue of standard errors is unrelated to the packages that are use
to \emph{report} your data. The way to go about this is to change the
standard errors in the model list to the appropriate standard errors
calculated by another package, in this case, \texttt{sandwich}.

\begin{Shaded}
\begin{Highlighting}[]
\NormalTok{model1 <-}\StringTok{ }\KeywordTok{lm}\NormalTok{(}\DataTypeTok{data =}\NormalTok{ GrowthDJ, }
\NormalTok{             gdpgrowth }\OperatorTok{~}\StringTok{ }\NormalTok{oil }\OperatorTok{+}\StringTok{ }\NormalTok{inter }\OperatorTok{+}\StringTok{ }\NormalTok{invest }\OperatorTok{+}\StringTok{ }\NormalTok{school)}

\KeywordTok{library}\NormalTok{(sandwich)}

\NormalTok{heterosk_vcov <-}\StringTok{ }\KeywordTok{vcovHC}\NormalTok{(model1, }\DataTypeTok{type =} \StringTok{"HC3"}\NormalTok{)}
\NormalTok{h_ac_vcov <-}\StringTok{ }\KeywordTok{vcovHAC}\NormalTok{(model1)}

\KeywordTok{stargazer}\NormalTok{(model1, }
          \KeywordTok{coeftest}\NormalTok{(model1, }\DataTypeTok{vcov =}\NormalTok{ heterosk_vcov), }
          \KeywordTok{coeftest}\NormalTok{(model1, }\DataTypeTok{vcov =} \KeywordTok{vcovHAC}\NormalTok{(model1) ))}
\end{Highlighting}
\end{Shaded}

\end{frame}

\begin{frame}{Standard errors}
\protect\hypertarget{standard-errors-1}{}

\def\arraystretch{0.7}

\begin{table}[!htbp] \centering 
  \caption{} 
  \label{} 
\scriptsize 
\begin{tabular}{@{\extracolsep{0pt}}lccc} 
\\[-1.8ex]\hline 
\hline \\[-1.8ex] 
 & \multicolumn{3}{c}{\textit{Dependent variable:}} \\ 
\cline{2-4} 
\\[-1.8ex] & gdpgrowth & \multicolumn{2}{c}{ } \\ 
\\[-1.8ex] & \textit{OLS} & \multicolumn{2}{c}{\textit{coefficient}} \\ 
 & \textit{} & \multicolumn{2}{c}{\textit{test}} \\ 
\\[-1.8ex] & (1) & (2) & (3)\\ 
\hline \\[-1.8ex] 
 oilno & $-$1.125$^{*}$ & $-$1.125$^{*}$ & $-$1.125$^{*}$ \\ 
  & (0.583) & (0.677) & (0.643) \\ 
  & & & \\ 
 interyes & 1.497$^{***}$ & 1.497$^{***}$ & 1.497$^{***}$ \\ 
  & (0.446) & (0.444) & (0.436) \\ 
  & & & \\ 
 oecdyes & $-$1.528$^{***}$ & $-$1.528$^{***}$ & $-$1.528$^{***}$ \\ 
  & (0.488) & (0.531) & (0.486) \\ 
  & & & \\ 
 invest & 0.099$^{***}$ & 0.099$^{***}$ & 0.099$^{***}$ \\ 
  & (0.026) & (0.035) & (0.032) \\ 
  & & & \\ 
 school & $-$0.026 & $-$0.026 & $-$0.026 \\ 
  & (0.062) & (0.076) & (0.069) \\ 
  & & & \\ 
 Constant & 2.711$^{***}$ & 2.711$^{***}$ & 2.711$^{***}$ \\ 
  & (0.646) & (0.817) & (0.748) \\ 
  & & & \\ 
\hline \\[-1.8ex] 
Observations & 115 &  &  \\ 
R$^{2}$ & 0.262 &  &  \\ 
Adjusted R$^{2}$ & 0.228 &  &  \\ 
\hline 
\hline \\[-1.8ex] 
\textit{Note:}  & \multicolumn{3}{r}{$^{*}$p$<$0.1; $^{**}$p$<$0.05; $^{***}$p$<$0.01} \\ 
\end{tabular} 
\end{table}

\end{frame}

\begin{frame}{Conclusion}
\protect\hypertarget{conclusion}{}

\begin{itemize}
\item
  Stargazer and sjPlot are two packages that can save you a lot of
  trouble (as long as you are not looking for correlation tables with
  stars).
\item
  Supplementary packages, such as xtable and hmisc could also help you
  in reporting the most common statistical analyses.
\item
  Thank you for your attention! Suggestions?
  \href{mailto:a.h.machielsen@uu.nl}{\nolinkurl{a.h.machielsen@uu.nl}}!
\end{itemize}

\end{frame}

\end{document}
